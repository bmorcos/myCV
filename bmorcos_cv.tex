% Adapted from https://github.com/Nadorrano/cv-friggeri-x
%
%%%%%%%%%%%%%%%%%%%%%%%%%%%%%%%%%%%%%%%%%
% cv-friggeri-x 1.0 (01/01/2016)
% XeLaTeX Template
%
% Based on:
% Friggeri Resume/CV
% Version 1.2 (3/5/15)
%
% Original author:
% Adrien Friggeri (adrien@friggeri.net)
% https://github.com/afriggeri/CV
%
% Modified by:
% Nadorrano
% https://github.com/Nadorrano/cv-friggeri-x
%
% License:
% MIT (https://opensource.org/licenses/MIT)
% CC BY-NC-SA 3.0 (http://creativecommons.org/licenses/by-nc-sa/3.0/)
%
% Important notes:
% This template needs to be compiled with XeLaTeX and the bibliography,
% if used, needs to be compiled with biber rather than bibtex.
%
%%%%%%%%%%%%%%%%%%%%%%%%%%%%%%%%%%%%%%%%%

\documentclass[a4paper,nocolors]{cv-friggeri-ben}
% Add `a4paper` to set a4 paper size
% Add `nocolors` to remove colors from the document
% Add `lightheader` to change the dark background of the header to white

% \usepackage{marvosym} % needed to print glyphs for email, cell phone etc.

\addbibresource{publications.bib} % Specify the bibliography
\urlstyle{same}  % URL style in bib


\begin{document}

\header{Ben}{Morcos}{curious neuromorphic engineer} % Your name and current job title/field

%%%%%%%%%%%%%%%%%%%
% SIDEBAR SECTION %
%%%%%%%%%%%%%%%%%%%

\begin{aside} % In the aside, each new line forces a line break
    \section{CONTACT}
    \hfill Waterloo
    \hfill Ontario, Canada
    ~
    \hfill +1.519.729.3223
    \hfill \href{mailto:morcos.ben@gmail.com}{morcos.ben@gmail.com}
    ~
    % \flogo \hfill \href{http://facebook.com/johnsmith}{fb://jsmith}
    % \tlogo \hfill \href{http://twitter.com/johnsmith}{@jsmith}
    % \llogo \hfill \href{http://linkedin.com/johnsmith}{in://johnsmith}
    % \vklogo \hfill \href{http://vk.com/johnsmith}{vk://johnsmith}
    % ~
    \hfill \href{https://github.com/bmorcos}{github.com/bmorcos}
    \section{LANGUAGES}
    English
    French
    \section{PROGRAMMING}
    Python
    C, C++
    OpenCL, CUDA
    VHDL, Verilog, HLS
    \LaTeX, shell, Tcl
    VBA
    \section{TOOLS}
    GitHub, GitLab
    Vivado, Quartus
    AutoCAD, SolidWorks
    MATLAB, Simulink
    \section{OPEN IP}
    \href{https://github.com/abr/zynq-axi-dna}{zynq-axi-dna}
    \href{https://github.com/abr/c5soc-ocl-id}{c5soc-ocl-id}
    \section{HOBBIES}
    Hiking \& Canoeing
    Climbing
    Woodworking
    Music
    Cooking
    Various sports
    Reading
\end{aside}


%%%%%%%%%%%%
% ABOUT ME %
%%%%%%%%%%%%

\section{ABOUT ME}
\vspace{-2pt}

\vspace{-5pt}
My BASc in Mechatronics Engineering gave me a broad foundation of skills which
allowed me to explore a variety of fields. However, developing hardware in the
context of neuroscience has me especially engaged and excited. Furthermore,
working with a plethora of leading PhD scientists allows me to learn everyday
and motivated me to pursue a master's degree.


%%%%%%%%%%%%%%%%%%%%%
% EDUCATION SECTION %
%%%%%%%%%%%%%%%%%%%%%

\section{EDUCATION}
\vspace{-2pt}

\begin{entrylist}

\entry
    {2017--2019}
    {MASc {\normalfont --- Computer Hardware Engineering}}
    {\\The University of Waterloo}
    {Working with FPGAs to develop \textit{neuromorphic} hardware to
    accelerate neural network computation with focus on flexibility and
    ease-of-use. The hardware is accessible by Python via the Nengo development
    framework and has run-time reconfigurability to support a wide range of
    neural networks with a static hardware design.}

\entry
    {2011--2016}
    {BASc {\normalfont --- Mechatronics Engineering, with distinction}}
    {\\The University of Waterloo}
    {The Mechatronics program covers a broad base of mechanical, electrical,
    computer, and system design engineering while my elective courses leaned
    towards philosophy, machine intelligence, and neuroscience. My capstone
    design project was a small-scale portable hydro-electric generator built
    from scratch.}

\end{entrylist}

%%%%%%%%%%%%%%%%%%%%%%%%%%%
% WORK EXPERIENCE SECTION %
%%%%%%%%%%%%%%%%%%%%%%%%%%%

\section{WORK EXPERIENCE}
\vspace{-2pt}

\begin{entrylist}

\entry
    {2016--Now}
    {Applied Brain Research}
    {Waterloo, ON}
    {\emph{Neuromorphic Engineer \& Lead Hardware Developer}
    \begin{itemize}
        \item Design flexible FPGA implementations to efficiently run dynamic
            neural networks.
        \item Maintaining a user-friendly interface layer between the FPGA and
            Python by extending the Nengo framework.
        \item Working in collaboration with leading scientists on
            neuro-robotics and various other computational neuroscience
            applications.
        \item Assisting with yearly \emph{Nengo Summer School} --- a two week
            in-depth workshop for international scholars to learn and use Nengo.
    \end{itemize}}

\entry
    {2014--2015}
    {Teledyne DALSA {\normalfont (co-op)}}
    {Waterloo, ON}
    {\emph{Mechanical Designer}
    \begin{itemize}
        \item Custom fixture designs starting with constraints and criteria and
            following through to vendor bids, manufacturing, and validation.
    \end{itemize}
    \emph{Sustaining Engineer}
    \begin{itemize}
        \item Optimizing and troubleshooting software and hardware by recreating
            manufacturing observations in a lab environment.
    \end{itemize}}

\entry
    {2014}
    {Toyota Motor Manufacturing Canada {\normalfont (co-op)}}
    {Cambridge, ON}
    {\emph{Quality Control Engineer for Lexus Hybrid group}
    \begin{itemize}
        \item Design of experiments to discover root cause as well as custom\\
            design and implementation of toolings to improve consistency.
        \item Coordinating interdepartmental operations and started \\
            new initiative to relate internal KPIs to user experience.
    \end{itemize}}

\end{entrylist}  % No page breaks inside environment, manually partitioning
\begin{entrylist}

\entry
    {2013}
    {Intellijoint Surgical {\normalfont \emph{formerly Avenir Medical} (co-op)}}
    {Waterloo, ON}
    {\emph{Medical Device Designer}
    \begin{itemize}
        \item Algorithm design and analysis, including test case development,
            with focus on image processing and feature extraction.
        \item Rapid prototyping of mechanical parts.
    \end{itemize}}

\entry
    {2012--2013}
    {IKO Industries {\normalfont (co-op)}}
    {Madoc, ON}
    {\emph{Mechanical Engineer}
    \begin{itemize}
        \item Improved throughput by 13\% with analysis and recommendation.
        \item Helped organize and analyze full process audit.
    \end{itemize}
    \emph{Electrical \& Systems Engineer}
    \begin{itemize}
        \item Optimized sensors, PLC, and HMI to save man-hours and improve
            consistency.
        \item Created a user-friendly database to track plant KPIs.
    \end{itemize}}

\end{entrylist}

%%%%%%%%%%%%%%%%%%%%%%%%
% PUBLICATIONS SECTION %
%%%%%%%%%%%%%%%%%%%%%%%%

\section{PUBLICATIONS}
\vspace{-4pt}

\printbibsection{mathesis}{~}
\printbibsection{inproceedings}{~}

%%%%%%%%%%%%%%%%
% VOLUNTEERING %
%%%%%%%%%%%%%%%%

\section{VOLUNTEER WORK}
\vspace{-2pt}
\begin{entrylist}

\entry
    {2017--Now}
    {The Foodbank of Waterloo Region}
    {Kitchener, ON}
    {Assisting at the distribution warehouse to sort food and keep track of
    local inventory. This is a fun, low mental effort, and social position that
    benefits the community --- everybody wins!}

\entry
    {2017}
    {Teaching Assistant}
    {Heterogeneous Architecture Summer School}
    {Assist with a one week workshop teaching students about computation using
    heterogeneous platforms (FPGA, GPU, CPU).}

\entry
    {2014--2015}
    {Federation Orientation Committee}
    {The University of Waterloo}
    {One of four volunteers responsible for planning Engineering Orientation
    Week 2015 for $\approx$2000 incoming students:
    \begin{itemize}
        \item Interview, hire, and manage a team of $\approx$400 volunteers.
        \item Obtain sponsorship and create a formal budget for the year.
        \item Work alongside numerous other entities within the University.
        \item Keep well documented records for continuous improvement.
    \end{itemize}}

\entry
    {2011--2015}
    {Campus Response Team}
    {The University of Waterloo}
    {Providing emergency first-aid for on-campus events.\\
    \emph{Operations Coordinator} --- 2014
    \begin{itemize}
        \item Manage and improve day-to-day and event operations.
    \end{itemize}
    \emph{Director of Training} --- 2012--2013
    \begin{itemize}
        \item Organize weekly training and termly first-aid competition.
    \end{itemize}}

\end{entrylist}

%%%%%%%%%%%%%%%%%%%%%%%%%%%
% CERTIFICATIONS & AWARDS %
%%%%%%%%%%%%%%%%%%%%%%%%%%%

\section{CERTIFICATIONS \& AWARDS}
\vspace{-2pt}

\begin{entrylist}

\entry
    {2009--Now}
    {Advanced Medical First Responder}
    {\\Canadian Ski Patrol System \& St. John Ambulance}
    {I used to volunteer as a ski patrol at Mt. Tremblant, QC and then I was
    a member of the Campus Response Team during my undergrad. I no longer
    actively provide first-aid but I still maintain my certification!}

\entry
    {2013}
    {NSERC Industrial Undergraduate Student Research Award}
    {Intellijoint Surgical}
    {\vspace{-11pt}}

\entry
    {2012}
    {Nominated as Co-op Student of the Year}
    {The University of Waterloo}
    {\vspace{-11pt}}

\entry
    {2011}
    {President's Scholarship}
    {The University of Waterloo}
    {\vspace{-11pt}}

\end{entrylist}

\end{document}
